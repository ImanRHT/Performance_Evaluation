\documentclass[12pt]{article}
\usepackage{geometry}
\geometry{
	letterpaper,
	left=20mm,
	right=20mm,
	top=25mm,
	bottom=30mm,
}
\usepackage{graphicx}
\usepackage{tikz}
\usepackage{tikzscale}
\usepackage{pgfplots}
\usepackage{amsmath,amssymb}
\usepackage{enumitem}
\usepackage{algorithm}
\usepackage{algorithmicx}
\usepackage{algpseudocode}
\usepackage{subcaption}
\usepackage{mathtools}
\usepackage{amsthm}
\usepackage{nccmath} %for fleqn
\makeatletter
\def\verbatim@font{\linespread{1}\normalfont\ttfamily}
\makeatother
\usepackage[breaklinks]{hyperref} 
\hypersetup{
	colorlinks   = true,
	citecolor    = blue
}
\hypersetup{linkcolor=blue}
\usepackage[T1]{fontenc}
\DeclarePairedDelimiter{\ceil}{\lceil}{\rceil}
\mathchardef\mhyphen="2D
\newcommand\makeset{\mathop{make\mhyphen set}}
\newcommand\findset{\mathop{find\mhyphen set}}
\newcommand*{\Comb}[2]{{}^{#1}C_{#2}}
\usepackage{mdframed}
\usepackage{fancyhdr} 
\fancyhf{}
\chead{\courseName}
\lhead{\thepage}  
\rhead{\homeworkName}
\renewcommand{\headrulewidth}{1pt}
\renewcommand{\footrulewidth}{0pt} 
\fancypagestyle{first}{% 
	\fancyhf{} % clear all header and footer fields 
	\chead{
		\small
		} % except the center 
	\renewcommand{\headrulewidth}{0pt} 
	\renewcommand{\footrulewidth}{0pt}
} 
\fancypagestyle{second}{% 
	\fancyhf{} % clear all header and footer fields 
	\chead{} % except the center 
	\renewcommand{\headrulewidth}{0pt} 
	\renewcommand{\footrulewidth}{0pt}
}
\pagestyle{fancy} 
	
\usepackage{xepersian}
\settextfont[
Scale=1.2,
Extension=.ttf, 
Path=../../common/fonts/,
BoldFont=*-BOLD
]{B-NAZANIN}
\setlatintextfont[Scale=1.1]{Times New Roman}
%\setdigitfont[Scale=1.4]{Times New Roman}
\setlength\parindent{0pt}

\newcommand{\courseName}{ارزیابی کارایی سیستم های کامپیوتری}
\newcommand{\courseSemester}{پاییز ۱۴۰۲}
\newcommand{\homeworkName}{تمرین دوم}
\newcommand{\homeworkDue}{موعد: }

\renewcommand{\baselinestretch}{1.5} 
\pgfplotsset{compat=1.18}
\begin{document}
	\graphicspath{{../../common/cover/},{img/}}
	\pagenumbering{gobble} 
	\thispagestyle{first}
%\newgeometry{top=2cm}
%\newcommand*{\SSS}{\includegraphics[scale=0.2]{ut}}%
%\newcommand*{\TTT}{\includegraphics[scale=0.2]{fanni}}%
\begin{mdframed}
\begin{minipage}[t]{0.2\textwidth}
	\centering
	\includegraphics[width=0.6\textwidth]{sharif} \\
%	\vspace{0.2cm}
	\homeworkName
\end{minipage}%
\begin{minipage}[b]{0.59\textwidth}
	\centering
	\courseName \\
	\courseSemester
\end{minipage}%
\begin{minipage}[t]{0.2\textwidth}
	\centering
	\includegraphics[width=0.6\textwidth]{sharif} \\
	\homeworkDue
\end{minipage}%
\end{mdframed}
%\restoregeometry

	\pagenumbering{arabic}  
	\input{Content/2.tex}
	\newpage


	\textbf{مسئله اول}
	\newline
	\textit{در یک رستوران فست‌فود دو تا از کارکنان سفارش‌ها را تحویل می‌دهند. نرخ سرویس صف اول دو برابر دیگری است (\lr{$\mu_1 = 2\mu_2$}). با فرض اینکه نرخ ورود به هر دو صف برابر (مقدار \lr{$\lambda$}) باشد، نسبت زمان انتظار در صف برای مشتریان صف اول به صف دوم را محاسبه کنید. (نرخ ورود، نرخ خروج و زمان سرویس از توزیع نمایی است)}
	\begin{center}
	    \includegraphics[width=0.15\textwidth]{Content/figure-1.png}
	\end{center}
    \bigskip
    
    
    \textbf{مسئله دوم}
    \newline
	\textit{در یک رستوران فست‌فود مشتری‌ها با نرخ \lr{$\lambda$} وارد می‌شوند. با احتمال \lr{p} به صف اول می‌پیوندند و با احتمال \lr{1 - p}
وارد صف بعدی می‌شوند. اگر نرخ سرویس صف اول برابر با \lr{$\mu_1$} و نرخ سرویس صف دوم برابر با \lr{$\mu_2$} باشد، مبانگین زمان انتظار هر مشتری برای دریافت سفارشاش چقدر خواهد بود؟ (نرخ ورود، نرخ خروج و زمان سرویس از توزیع نمایی است)}
	\begin{center}
	    \includegraphics[width=0.5\textwidth]{Content/figure-2.png}
	\end{center}
    \bigskip
    
    
    \textbf{مسئله سوم}
    \newline
	\textit{فرض کنید نرخ ورود، زمان سرویس و نرخ سرویس همه از توزیع نمایی باشد. میانگین زمان پاسخ (\lr{$E[T]$}) برای شبکه
نشان داده شده چقدر است؟}
	\begin{center}
	    \includegraphics[width=0.7\textwidth]{Content/figure-3.png}
	\end{center}
    \bigskip
    
    
    \textbf{مسئله چهارم}
    \newline
	\textit{با توجه به شکل زیر، میانگین زمان پاسخ را برای کارهای کلاس نوع اول و میانگین زمان پاسخ را برای کارهای کلاس
نوع دوم محاسبه کنید. نرخ‌های ورود و خروج و زمان پردازش همه از توزیع نمایی هستند.}
	\begin{center}
	    \includegraphics[width=0.6\textwidth]{Content/figure-4.png}
	\end{center}
    \bigskip
%	{
%		\small
%		\bibliographystyle{ieeetr-fa}
%		\bibliography{MyReferences}
%	}
\end{document}