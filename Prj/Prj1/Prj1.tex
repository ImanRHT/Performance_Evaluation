\documentclass[12pt]{article}
\usepackage{geometry}
\geometry{
	letterpaper,
	left=20mm,
	right=20mm,
	top=25mm,
	bottom=30mm,
}
\usepackage{graphicx}
\usepackage{tikz}
\usepackage{tikzscale}
\usepackage{pgfplots}
\usepackage{amsmath,amssymb}
\usepackage{enumitem}
\usepackage{algorithm}
\usepackage{algorithmicx}
\usepackage{algpseudocode}
\usepackage{subcaption}
\usepackage{mathtools}
\usepackage{amsthm}
\usepackage{nccmath} %for fleqn
\makeatletter
\def\verbatim@font{\linespread{1}\normalfont\ttfamily}
\makeatother
\usepackage[breaklinks]{hyperref} 
\hypersetup{
	colorlinks   = true,
	citecolor    = blue
}
\hypersetup{linkcolor=blue}
\usepackage[T1]{fontenc}
\DeclarePairedDelimiter{\ceil}{\lceil}{\rceil}
\mathchardef\mhyphen="2D
\newcommand\makeset{\mathop{make\mhyphen set}}
\newcommand\findset{\mathop{find\mhyphen set}}
\newcommand*{\Comb}[2]{{}^{#1}C_{#2}}
\usepackage{mdframed}
\usepackage{fancyhdr} 
\fancyhf{}
\chead{\courseName}
\lhead{\thepage}  
\rhead{\homeworkName}
\renewcommand{\headrulewidth}{1pt}
\renewcommand{\footrulewidth}{0pt} 
\fancypagestyle{first}{% 
	\fancyhf{} % clear all header and footer fields 
	\chead{
		\small
		} % except the center 
	\renewcommand{\headrulewidth}{0pt} 
	\renewcommand{\footrulewidth}{0pt}
} 
\fancypagestyle{second}{% 
	\fancyhf{} % clear all header and footer fields 
	\chead{} % except the center 
	\renewcommand{\headrulewidth}{0pt} 
	\renewcommand{\footrulewidth}{0pt}
}
\pagestyle{fancy} 
	
\usepackage{xepersian}
\settextfont[
Scale=1.2,
Extension=.ttf, 
Path=../../common/fonts/,
BoldFont=*-BOLD
]{B-NAZANIN}
\setlatintextfont[Scale=1.1]{Times New Roman}
%\setdigitfont[Scale=1.4]{Times New Roman}
\setlength\parindent{0pt}

\newcommand{\courseName}{ارزیابی کارایی سیستم های کامپیوتری}
\newcommand{\courseSemester}{پاییز ۱۴۰۲}
\newcommand{\homeworkName}{پروژه اول}
\newcommand{\homeworkDue}{موعد: }

\renewcommand{\baselinestretch}{1.5} 

\begin{document}
	\graphicspath{{../../common/cover/},{img/}}
	\pagenumbering{gobble} 
	\thispagestyle{first}
%\newgeometry{top=2cm}
%\newcommand*{\SSS}{\includegraphics[scale=0.2]{ut}}%
%\newcommand*{\TTT}{\includegraphics[scale=0.2]{fanni}}%
\begin{mdframed}
\begin{minipage}[t]{0.2\textwidth}
	\centering
	\includegraphics[width=0.6\textwidth]{sharif} \\
%	\vspace{0.2cm}
	\homeworkName
\end{minipage}%
\begin{minipage}[b]{0.59\textwidth}
	\centering
	\courseName \\
	\courseSemester
\end{minipage}%
\begin{minipage}[t]{0.2\textwidth}
	\centering
	\includegraphics[width=0.6\textwidth]{sharif} \\
	\homeworkDue
\end{minipage}%
\end{mdframed}
%\restoregeometry

	\pagenumbering{arabic}  
	در پروژه اول، هر دانشجو باید سوال‌هایی از کتاب مرجع را که به ایشان انتصاب داده‌شده است، انجام دهد. فایل اکسل سوالات انتصاب داده‌شده به دانشجویان در ایلرن قرار دارد.

\textbf{در ارسال تمرین موارد زیر رعایت شود:}

\begin{itemize}
	\item[-]
	 بیان مسئله و پاسخ سوال به طور دقیق و کامل به زبان فارسی نوشته شود.

	\item[-]	
	 نوشته شما باید نشان‌دهد مسئله و جواب را فهمیده‌اید.
	
	\item[-]	
	اگر بتوانید توضیحات اضافی به پاسخ‌تان بیفزایید تا آن را قابل‌ درک‌تر کنید، نمره اضافی دریافت خواهید کرد.
\end{itemize}

\textbf{نکاتی که باید توجه داشته باشید:}

\begin{itemize}
	\item[الف)]
	مهلت ارسال در سربرگ تمرین همچنین در ایلرن درج شده‌است.
	\item[ب)]
	کلیه تمرینات (غیر از تمرینات سرکلاسی) از طریق ایلرن دریافت می‌شوند و دیگر شیوه‌های ارسال تمرین پذیرفته نیست.
	\item[ج)]
	قالب تمرینات به‌صورت \lr{\LaTeX} و تنها در \lr{Template} تمرینات مورد پذیرش است. (\lr{Template} در ایلرن در دسترس است.)
	\item[د)]	
	فایل تمرین ارسالی باید شامل فایل‌های مورد نیاز به جهت اجرای فایل \lr{\LaTeX} به همراه \lr{PDF} باشد. نام فايل را به‌صورت زير انتخاب كنيد:

\begin{latin} \begin{center}
		\texttt{HW1\_Student\#\_Name}
	\end{center}
\end{latin}

	\item[ه)]
	ارسال با تأخیر تمرین، تنها تا سه‌روز پس از مهلت تمرین امکان‌پذیر بوده و به ازای هر روز 5 درصد کسر نمره خواهد داشت. پس از گذشت این مهلت، امکان ارسال تمرین میسر نیست.
\end{itemize}

	\newpage

	\textit{
	هدف از این تمرین شبیه‌سازی یک سیستم صف \lr{$M/M/1/K$} (یک صف و یک پردازنده) با نرخ سرویس‌دهي \lr{$\mu$} برای پردازنده و ظرفیت محدود \lr{K = 14} برای صف است. خط مشي سرویس‌دهي به صورت \lr{$FCFS$} است. هر فردی که وارد سیستم مي‌شود فقط برای مدت زمان مشخصي تا دریافت سرویس مي‌تواند منتظر بماند، این مدت زمان را با متغیر تصادفي \lr{$\theta$} نشان مي‌دهیم. بنابراین هر
فرد پس از گذشت مدت زمان \lr{$\theta$}، در صورت عدم دریافت سرویس، صف را ترک خواهد کرد.
	}
	\begin{itemize}
	\item[-]
زمان انتظار را تا لحظه‌ی شروع سرویس‌دهي در نظر بگیرید. دقت کنید پس از این که یک مشتری شروع به دریافت
سرویس کند، حتي با رسیدن موعد خود، صف را ترک نخواهد کرد.
	\item[-]	
	تابع توزیع زمان انتظار (\lr{$\theta$}) را در دو حالت ثابت و نمایي در نظر بگیرید. سپس با استفاده از روش شبیه‌سازی، برای هر یک
از حالت‌ها: 
    \begin{itemize}
	\item[-]
نمودار احتمال خارج شدن (\lr{$P_d$}) را نسبت به تغییرات نرخ ورودی \lr{$\lambda$} (در بازه \lr{$[0.1-20]$}
با میزان پرش‌ \lr{$0.1$})، با میانگین
زمان انتظار \lr{$\theta$} بدست آورید.
	\item[-]	
نمودار احتمال بلوکه شدن (\lr{$P_b$}) را نسبت به تغییرات نرخ ورودی \lr{$\lambda$} (در بازه \lr{$[0.1-20]$}
با میزان پرش‌ \lr{$0.1$})، با میانگین
زمان انتظار \lr{$\theta$} بدست آورید.
	 \item[-]	
نمودار متوسط تعداد مشتری‌های داخل صف (\lr{$N_c$}) را نسبت به تغییرات نرخ ورودی \lr{$\lambda$} (در بازه \lr{$[0.1-20]$}
با میزان پرش‌ \lr{$0.1$})، با میانگین
زمان انتظار \lr{$\theta$} بدست آورید.
    \end{itemize}
    \end{itemize}
    \bigskip
    
    \newpage
    \textbf{روش تحلیلی}
    \newline
    \textit{به‌ کمک روابط زیر خطای روش شبیه‌سازی خود را محاسبه کنید.}
    \begin{itemize}
	\item[-]
	احتمال اینکه \lr{$n$} نفر درون صف \lr{$M/M/1/K$} با موعد انتظار باشند:
    \lr{\[
	P_n =
	\begin{cases}
	    P_0(\frac{\lambda}{\mu}),
		&\quad\text{n = 1} \\
		P_0\lambda^n\frac{\varphi_{n-1}(\mu)}{(n-1)!},
		&\quad\text{n > 1}
	\end{cases}
	\]}
    
    \item[-]
	که مقدار تابع \lr{$\varphi_n(\mu)$} با توجه به توزیع زمان انتظار به‌دست مي‌آید. برای حالت موعد دارای توزیع نمایي:
    \lr{\[
	\varphi_n(\mu) = \frac{n!}{\prod_{i=0}^{n} (\mu + i/\bar\theta)}
	\]}
    \item[-]
	برای حالت موعد ثابت:
    \lr{\[
	\varphi_n(\mu) = \frac{n!}{\mu^{n+1}}(1 - e^{-\mu\bar\theta}\sum_{i=0}^{n-1} \frac{(\mu\bar\theta)^i}{i!})
	\]}
    \item[-]
	اگر طول صف برابر \lr{$K$} باشد:
    \lr{\[
    \sum_{i=0}^{K} P_i = 1
	\]}
    \item[-]
	حال از روابط زیر برای محاسبه مقادیر \lr{$P_b$} و \lr{$P_d$} استفاده کنید:
    \lr{\[
    P_b = P_K
    \]
    \[
    P_d + P_b = 1 - \frac{\mu}{\lambda}(\sum_{i=1}^{K} P_i) = 1 - \frac{\mu}{\lambda}(1 - P_0)
	\]}
    \item[-]
	برای محاسبه‌ی \lr{$N_c$} :
    \lr{\[
    N_c = \sum_{i=1}^{K} (i.P_i)
    \]}
	\end{itemize}
	
	\newpage
    \textbf{نکات و سوالات متداول}
    \begin{itemize}
	\item[-]
	طول صف نشان‌دهنده‌ی تمامي مشتری‌های حاضر در سیستم مي‌باشد. به طور مثال در این تمرین اگر \lr{$13$} مشتری در صف
منتظر باشند و یک مشتری در صف در حال سرویس‌دهي باشد، مشتری‌های جدید بلاک خواهند شد. 
	\item[-]
برای بدست آوردن یک عدد تصادفي با توزیع نمایي مي‌توان از رابطه زیر کمک گرفت. در این رابطه \lr{$x$} یک عدد تصادفي با
توزیع یکنواخت \lr{$(Uniform)$} در بازه \lr{$[0,1)$} مي‌باشد. در این رابطه \lr{$\lambda$} همان نرخ ورود مشتری به سیستم است. 
    \[
    y = - \frac{\ln(1 - x)}{\lambda}
    \]
    \item[-]
برای کمینه کردن خطای نتایج شبیه‌سازی توصیه مي‌شود تعداد مشتری‌های ورودی به سیستم \lr{$10^7$} یا \lr{$10^8$} در نظر گرفته
شود. برای هر مشتری سرانجام یکي از سه وضعیت سرویس‌گرفتن، بلاک شدن (وقتي صف پر است) و ترک صف (زمان  رسیدن موعد) اتفاق خواهد افتاد.
	\item[-]
ورود مشتری‌ها به سیستم یک فرآیند پواسون با پارامتر \lr{$\lambda$} است. در نتیجه، زمان بین ورود مشتری‌ها مستقل و از توزیع
نمایي با پارامتر \lr{$\lambda$} پیروی مي‌کند. 
	\end{itemize}
	
	\textbf{تست برنامه}
	\newline
    \textit{به منظور بررسي صحت کد ارسالي، موارد زیر را حتما در مورد کد ارسالي رعایت کنید.}
	\begin{itemize}
	\item[-]
در کنار کد خود یک \lr{$Makefile$} قرار دهید. برای کامپایل و اجرای کد شما دستورات زیر استفاده خواهد شد.
    \begin{latin}
		\texttt{make}\\
		\texttt{make run}
	\end{latin}
    \item[-]
    برنامه شبیه‌سازی ارسال‌شده توسط شما باید مقدار پارامتر \lr{$\theta$} و \lr{$\mu$} را به عنوان ورودی بگیرد و مقدار \lr{$P_d$} و \lr{$P_b$} برای حالت‌های
شبیه‌سازی و تحلیلي فقط برای \lr{$\lambda$} های ۵، ۱۰ و ۱۵ و زمان انتطار ثابت را در یک فایل متني چاپ کند. مقادیر پارامترهای ورودی در فایل \lr{$parameters.conf$} در کنار \lr{$Makefile$} قرار مي‌گیرد. یک فایل نمونه در کنار پروژه قرار گرفته‌است.
    \end{itemize}
    
    \newpage
    \textbf{نکات پایانی}
    \begin{itemize}
	\item[-]
	شبیه‌سازی مي‌تواند با استفاده از زبان‌های برنامه‌نویسي \lr{$C++$} ،\lr{$C$}، جاوا یا پایتون انجام شود.
	\item[-]
	پروژه‌های ارسالي باید شامل کد استفاده‌شده و نتایج گرفته‌شده در قالب فایل \lr{$Excel$} با پارامترهای گفته‌شده در بالا
باشد.
    \item[-]
    قالب فایل‌های \lr{$Excel$} در کل شامل شش نمودار هستند؛ سه نمودار برای موعد ثابت و سه نمودار برای موعد نمایي که هر
کدام شامل نتایج \lr{$P_d$} و \lr{$P_b$} و \lr{$N_c$} با پارامترهای 3 = \lr{$\theta$} و 1 = \lr{$\mu$} ميباشند. هر کدام از نمودارها شامل دو منحني است که
مقدار احتمالات به‌دست آمده از روش شبیه‌سازی و تحلیلي به‌ازای نرخ ورودی داده شده در مسأله را نشان مي‌دهد.
	\end{itemize}
	
\end{document}